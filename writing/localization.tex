\paragraph{Localization Daemon}
Because sensor data can be noisy, kinematics models will fail to fully account for wheel slippage, and getting lost is not an option, a particle filter is used in order to refine the pose of the vehicle after every movement.  The inital particle swarm is defined according to the rules of the competition.  Since the filter's pose is relative to the sensor position, which is mounted at the rear of the robot, the robot being centered on one of two particular positions with a random orientation means we start off with particles distributed about two circles.  Each particle will have specific orientations depending on its position on the circle.  The particle filter then proceeds by first evaluating each particle's fitness.  It does this by comparing the LIDAR range data to the known map of the arena, assuming the pose of the given particle.  The fitness of all particles is renormalized to create probabilities.  Finally, the particles are ordered by probability and resampled.  Those particles with higher probabilities will be chosen more often than those with lower probabilities.
